\begin{enumerate}[label=\alph*)]
  \item 
برای اولین قسمت از پروتکل 
\lr{diffie-helman}
برای ساخت و تبادل کلید استفاده می‌کنیم.
\begin{latin}
  $C \rightarrow KDC: IP_S$\\
  $KDC \rightarrow C: K_{C,S}$\\
  $C \rightarrow S: E(K_{C,S},\alpha || q || Y_C = (\alpha^{X_C} mod q))$\\
  $S \rightarrow C: E(K_{C,S}, Y_S = (\alpha^{X_S} mod q))$\\
  $K' = Y_C ^ {X_S} modq = Y_S ^ {X_C} modq$
\end{latin}
حال عملا با دیفی هلمن توانستیم 
کلیدی ایجاد کنیم که KDC 
از آن اطلاعی ندارد اما طرفین آن را می‌دانند.
\item 
در صورتی که KDC 
توان تغییر ترافیک را داشته باشد می‌تواند حمله Man-In-The-Middle
را پیاده‌سازی کند و عملا خود را برای S جای C و برای C جای S جا بزند.
\begin{latin}
  $C \rightarrow KDC: IP_S$\\
  $KDC \rightarrow C: K_{C,S}$\\
  $C \rightarrow S: E(K_{C,S},\alpha || q || Y_C = (\alpha^{X_C} mod q))$\\
  $\text{KDC Intercept -- }KDC \rightarrow S: E(K_{C,S},\alpha || q || Y_C' = (\alpha^{X_{KDC}} mod q))$\\
  $S \rightarrow C: E(K_{C,S}, Y_S = (\alpha^{X_S} mod q))$\\
  $\text{KDC Intercept -- }KDC \rightarrow C: E(K_{C,S},\alpha || q || Y_S' = (\alpha^{X_{KDC}} mod q))$\\
\end{latin}


\item 
می‌دانیم 
KDC
ها
با یکدیگر امکان تبانی ندارند پس منطقا از کلید‌های یکدیگر نیز خبر ندارند.
تنها کاری که لازم است انجام دهیم این است که K'
را با هر دو کلید رمز کنیم.
\begin{latin}
  $C \rightarrow S: E(K_{C,S,2},E(K_{C,S,1},K'))$\\
\end{latin}
حال در صورتی که یکی از KDC
ها 
پیام را باز کند، نمی‌تواند متوجه پیام شود چون 
\lr{$KDC_1$}
نمی‌تواند پیام بیرونی را باز کند و 
\lr{$KDC_2$}
نیز نمی‌تواند پیام داخلی را باز کند.
\end{enumerate}